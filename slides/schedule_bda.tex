\documentclass{mjandrews.notes.nomp}
\usepackage{booktabs}
\usepackage{a4wide}
\usepackage{mathpazo, eulervm}
\usepackage{lipsum}
\title{Bayesian Data Analysis Workshops}
\author{Mark Andrews \\ Department of Psychology \\ Nottingham-Trent University}
\date{}


\begin{document}
\maketitle

\section{Introduction}

The aim of this pair of workshops is to give a theoretical and practical
introduction to Bayesian data analysis. The first workshop will primarily deal
with the theoretical introduction, with the second workshop primarily dealing
with how to do Bayesian data analysis in practice, using real world data sets
and a wide range of computational tools available in the R programming
language.  Throughout both days, our focus will be largely on a set of widely
used and practically useful statistical models, particularly the general linear
models, generalized linear models, and their multilevel counterparts. 

In what follows, we provide a schedule for each workshop and brief overview of
topics covered in each one.

\section{Workshop 1}

\begin{center}
	\begin{tabular}{rp{0.7\textwidth}}
		9:30am & Historical \& general overview \\
		11:30am & Bayes's theorem and statistical inference\\
		12:30pm & \emph{Break} \\
		1:30pm & Analytically tractable Bayesian models \\
		3:30pm & Bayesian inference using Markov Chain Monte Carlo \\
		5:30pm & Close  \\
\end{tabular}
\end{center}


\subsection*{Historical, philosophical, \& general overview}

Bayesian data analysis is a general approach to statistical data analysis and
is often contrasted with the more familiar \emph{classical} or
\emph{frequentist} approach to data analysis.   The aim of this section to
provide a general overview of what Bayesian methods are, how they differ from
the more familiar classical methods, and how and why they have risen in
prominence in recent decades.

\subsection*{Bayes's theorem and statistical inference}

In this section, we will describe a simple example of Bayesian inference and
show how this can be extended to provide a general means for inference in
statistical models. The aim here is to understand the fundamentals of Bayesian
statistical modelling using some simple and intuitive examples.

\subsection*{Analytically tractable Bayesian models}

In some probabilistic models, such as linear models with Normally distributed
errors, Bayesian inference is analytical tractable: There are closed form
solutions for inference of parameters, predictions of unobserved data, and
model comparison.  In this section, we concentrate on some example of these
models, and derive the solutions for inference and predictions. In this
section, we will also focus on the general topic of conjugate priors, and how
to do Bayesian hypothesis testing and model comparison using Bayes factors.

\subsection*{Bayesian models using Markov Chain Monte Carlo}

In general, inference and prediction in Bayesian models is not analytically
tractable. In these situations, we use Monte Carlo methods, particularly Markov
Chain Monte Carlo (MCMC) methods, to make inferences and predictions on the
basis of samples. In this section, we will consider why it is necessary to use
Monte Carlo methods and MCMC methods, and we will describe the methods such as
rejection sampling, importance sampling, Gibbs sampling, Metropolis-Hastings
sampling. We will also provide an overview of probabilistic modelling languages
that allow us to easily set up and use MCMC sampling methods.

\section{Workshop 2}

\begin{center}
	\begin{tabular}{rp{0.7\textwidth}}
		9:30am & Bayesian data analysis using R\\
		10:30am & Bayesian linear models \\
		12:30pm & \emph{Break} \\
		1:30pm & Bayesian generalized linear models \\
		3:30pm & Bayesian multilevel models \\
		5:30pm & Close  \\
\end{tabular}
\end{center}

\subsection{Bayesian data analysis with R}

We will begin with an overview of how to R and RStudio for Bayesian data
analysis. We will focus on the main and popular Bayesian R packages, and show
how R can be used with probabilistic modelling languages such JAGS.

\subsection{Bayesian linear models}

In this section, we will focus on inference, prediction, and model comparison
in linear regression and general linear models. Using this set of examples, we
will be able to easily see the main differences, as well as the similarities,
between Bayesian and classical approaches to data analysis.

\subsection{Bayesian generalized linear models}

Generalized linear models -- logistic regression, Poisson regression, ordinal
logistic regression, etc. -- require MCMC methods for inference and prediction.
In this section, we will primarily focus on how to use JAGS as a flexible and
powerful approach to Bayesian modelling with MCMC.

\subsection{Bayesian multilevel models}

In this section, we will consider the multilevel extensions of linear and
generalized linear models. In this section, we will again use JAGS, as well as
some specially made packages for Bayesian multilevel modelling.


\section*{About the presenter}

Dr.\ Mark Andrews is a senior lecturer in experimental psychology in the
Department of Psychology, Nottingham Trent University. His research focuses on
the application of Bayesian methods to the study of human cognition. For the
past few years, he has been funded by the ESRC to provide workshops on Bayesian
Data analysis to researchers in the social sciences, see
\[
\texttt{http://www.priorexposure.org.uk}
\]



\section*{Preparing for the workshop}

You should bring your own laptop computer with R, and R studio, JAGS and rjags
pre-installed. Instructions for setting up your laptop can be found here:
\[
\texttt{http://www.priorexposure.org.uk/software}
\]


\end{document}
