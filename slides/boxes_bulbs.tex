\documentclass{slides}
\usepackage{xspace,tikz,graphicx}

\title{The ``Boxes and Bulbs'' Problem}
\author[Andrews]{Mark Andrews}
\date{}

\begin{document}
{
	\begin{frame}
		\titlepage
	\end{frame}
}


\begin{frame}
\frametitle{The problem}

The ``Boxes and Bulbs'' problem\footnote{This problem is taken from \emph{Schaum's Outlines: Probability} (2nd ed., 2000), pages 87-88.} is as follows:
\begin{quotation}
{\itshape
Box A has 10 lightbulbs, of which 4 are defective.
Box B has 6 lightbulbs, of which 1 is defective.
Box C has 8 lightbulbs, of which 3 are defective.
\\
I) If we randomly choose a box, and then randomly choose a lightbulb from that box, what is the probability that we will choose a non-defective bulb?
II) If we do choose a nondefective bulb, what is the probability it came from Box C? 
}
\end{quotation}
\end{frame}
%
%\section{The solution}
\begin{frame}
\frametitle{The solution}

First, note that the solution requires the calculation of two probabilities:
\[
\Prob{\text{Bulb}=\text{Working}} \quad\text{and}\quad \Prob{\text{Box}=C\given\text{Bulb}=\text{Working}}.
\]
These are the answers to Part I and Part II, respectively. Second, if you
correctly work out the first probability, i.e.
$\Prob{\text{Bulb}=\text{Working}}$, then will have all the information
necessary to obtain the second probability, i.e.
$\Prob{\text{Box}=C\given\text{Bulb}=\text{Working}}$. 
\end{frame}

\begin{frame}
	\frametitle{The generative model}
% Set the overall layout of the tree
\tikzstyle{level 1}=[level distance=4.0cm, sibling distance=3.0cm]
\tikzstyle{level 2}=[level distance=5.5cm, sibling distance=1.5cm]

% Define styles for bags and leafs
\tikzstyle{branch} = [text width=4em, text centered]
\tikzstyle{leaf} = [circle, minimum width=3pt,fill, inner sep=0pt]

% The sloped option gives rotated edge labels. Personally
% I find sloped labels a bit difficult to read. Remove the sloped options
% to get horizontal labels. 
\begin{figure}
\begin{tikzpicture}[grow=right, sloped, scale=0.6, every node/.style={scale=0.6}]
%\tikzstyle{every node}=[font=\footnotesize]
\node[branch] {}
    child {
        node[branch] {Box C}        
           child {
                node[leaf, label=right:
                    {$\Prob{\text{C},\text{Defective}} = \frac{1}{3}\times\frac{3}{8}\approx.125$}] {}
                edge from parent
                node[below] {$\Prob{\text{Defective}\given\text{C}}=\frac{3}{8}$}
            } 
	    child {
                node[leaf, label=right:
                    {$\Prob{\text{C},\text{Working}} = \frac{1}{3}\times\frac{5}{8}\approx.208$}] {}
                edge from parent
                node[above] {$\Prob{\text{Working}\given\text{C}}=\frac{5}{8}$}
            }
            edge from parent 
            node[above] {$\Prob{C}=\frac{1}{3}$}
    }
    child {
        node[branch] {Box B}        
            child {
                node[leaf, label=right:
                    {$\Prob{\text{B},\text{Defective}} = \frac{1}{3}\times\frac{1}{6}\approx.056$}] {}
                edge from parent
                node[below] {$\Prob{\text{Defective}\given\text{B}}=\frac{1}{6}$}
            }
           child {
                node[leaf, label=right:
                    {$\Prob{\text{B},\text{Working}} = \frac{1}{3}\times\frac{5}{6}\approx.278$}] {}
                edge from parent
                node[above] {$\Prob{\text{Working}\given\text{B}}=\frac{5}{6}$}
            } edge from parent 
            node[above] {$\Prob{B}=\frac{1}{3}$}
    }
    child {
        node[branch] {Box A}        
           child {
                node[leaf, label=right:
                    {$\Prob{\text{A},\text{Defective}} = \frac{1}{3}\times\frac{4}{10}\approx.133$}] {}
                edge from parent
                node[below] {$\Prob{\text{Defective}\given\text{A}}=\frac{4}{10}$}
            } 	
           child {
                node[leaf, label=right:
                    {$\Prob{\text{A},\text{Working}} = \frac{1}{3}\times\frac{6}{10}\approx.2$}] {}
                edge from parent
                node[above] {$\Prob{\text{Working}\given\text{A}}=\frac{6}{10}$}
            }
            edge from parent 
            node[above] {$\Prob{A}=\frac{1}{3}$}
    }
    ;
\end{tikzpicture}
\end{figure}

\end{frame}

\begin{frame}
	\begin{itemize}
	\item This tree structure gives you all the information you need to answer the questions. Note that on the right are given all the joint probabilities, which can be arranged to a joint probability table as follows:
\begin{center}
\begin{tabular}{rcc}
& Working & Defective \\
Box A & $.2$ & $.133$ \\
Box B & $.278$ & $.056$ \\
Box C & $.208$ & $.125$
\end{tabular}
\end{center}
\item Now it is simple to see the overall probability of choosing a working bulb. It is probability of choosing Box A and choosing a working bulb \emph{or} choosing Box B and choosing a working bulb \emph{or} choosing Box C and choosing a working bulb. This is the sum of the first column, which is $.2+.278+.208\approx.686$. 
\item Likewise, to get the probability of choosing Box C \emph{given} that we have chosen a working bulb, we see what proportion of the total probability of choosing a working bulb, i.e. $.686$ is from when we choose Box C and a working bulb, i.e. $.208$. In other words, it is $.208/.686\approx .304$. 
%
\end{itemize}
\end{frame}

\begin{frame}
	\frametitle{Using probability rules}
	\begin{itemize}
	\item The first question asks us to calculate $\Prob{\text{Bulb}=\text{Working}}$. We know that this is equal to 
\begin{align*}
\Prob{\text{Working}} &= \Prob{\text{Working},A} + \Prob{\text{Working}, B} + \Prob{\text{Working},C},\\
&= .2 + .278 + .208.\\
&\approx .686.
\end{align*}
\item When asked for $\Prob{C\given\text{Working}}$ we use the rule of conditional probability being the joint probability divided by the marginal probability, i.e. 
\begin{align*}
\Prob{C\given\text{Working}} &= \frac{\Prob{\text{Working},C}}{\Prob{\text{Working}}},\\
&= \frac{.208}{.686},\\
&\approx .304.
\end{align*}

	\end{itemize}
\end{frame}

%\subsection{Using probability rules}
%
%The probabilities just mentioned can all be calculated using rules of
%probability. In fact, this is precisely what is being done using the tree
%structure to get the joint probability table and then summing and dividing,
%etc., as appropriate. It is important to be able to see this explicitly.  
%
%The first question asks us to calculate $\Prob{\text{Bulb}=\text{Working}}$. We know that this is equal to 
%\begin{align*}
%\Prob{\text{Working}} &= \Prob{\text{Working},A} + \Prob{\text{Working}, B} + \Prob{\text{Working},C},\\
%&= .2 + .278 + .208.\\
%&\approx .686.
%\end{align*}
%In fact, that is precisely what we did above when we summed the first column of the joint probability table. We also know that any given joint probability such as $\Prob{\text{Working},C}$ is calculated by 
%\begin{align*}
%\Prob{\text{Working},C} &= \Prob{\text{Working}\given C} \Prob{C},\\
%&=  \frac{5}{8} \times \frac{1}{3},\\
%&\approx .208.
%\end{align*}
%Notice that this is exactly what we are doing in the tree structure calculations. Finally, when asked for $\Prob{C\given\text{Working}}$ we use the rule of conditional probability being the joint probability divided by the marginal probability, i.e. 
%\begin{align*}
%\Prob{C\given\text{Working}} &= \frac{\Prob{\text{Working},C}}{\Prob{\text{Working}}},\\
%&= \frac{.208}{.686},\\
%&\approx .304.
%\end{align*}
%
\end{document}

