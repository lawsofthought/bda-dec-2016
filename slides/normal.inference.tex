\documentclass{slides}

\title[Bayesian Inference]{Bayesian Inference in Normal Models}
\author[Andrews]{Mark Andrews}
\date{}

\begin{document}
{
	\begin{frame}
		\titlepage
	\end{frame}
}


\begin{frame}
	\frametitle{Inference in normal distributions}
	\begin{itemize}
		\item The univariate normal, or Gaussian, probability distribution with mean $\mu$ and variance $\sigma^2$ has the density
			\[
				\Prob{x\given\mu,\sigma} = \frac{1}{\sqrt{2\pi\sigma^2}}e^{-\frac{(x-\mu)^2}{2\sigma^2}}.
			\]
		\item A random variable $x$ that is normally distributed with mean $\mu$ and variance $\sigma^2$ can be written
			\[
				x \sim \textrm{N}(\mu, \sigma^2).
			\]

	\end{itemize}
\end{frame}

\begin{frame}
	\frametitle{Inference in normal distributions}
	\begin{itemize}
		\item A common elementary data analysis problem is when we
			observe $n$ iid samples from $\textrm{N}(\mu, \sigma^2)$, where
			$\mu$ and $\sigma^2$ are unknown, and we aim to infer
			$\mu$.

		\item In other words, we observe
			\[D = x_1, x_2 \cdots x_n,\]
			where we assume
			\[x_i \sim \textrm{N}(\mu, \sigma^2),\quad\text{for $i \in 1,2\cdots n$},\]
		      where both $\mu$ and $\sigma^2$ are unknown, and we aim to infer the value of $\mu$. 
		\item In other words, we aim to determine the posterior distribution
		      \[
			\Prob{\mu \given D}.
		      \]
	\end{itemize}
\end{frame}


\begin{frame}
	\frametitle{Likelihood of $\mu$ and $\sigma^2$}
	\begin{itemize}
		\item The likelihood of $\mu$ and $\sigma^2$ given $D$ is 
			\begin{align*}
				\Prob{x_1, x_2 \cdots x_n \given \mu, \sigma^2} &= \prod_{i=1}^n \Prob{x_i \given \mu, \sigma^2},\\
				&= \prod_{i=1}^n \frac{1}{\sqrt{2\pi\sigma^2}}e^{-\frac{(x_i-\mu)^2}{2\sigma^2}},\\
				&= \left(\frac{1}{\sqrt{2\pi\sigma^2}}\right)^n e^{\left(-\tfrac{1}{2\sigma^2}\sum_{i=1}^{n}(x_i - \mu)^2\right)},\\
				&\propto \sigma^{-n} e^{\left( -\tfrac{1}{2\sigma^2}\left[n s^2 + n(\bar{x}-\mu)^2\right] \right)},
			\end{align*}
			with \[
				\bar{x} \doteq \tfrac{1}{n} \sum_{i=1}^n x_i, \quad s^2 \doteq \tfrac{1}{n} \sum_{i=1}^n (x_i - \bar{x})^2.
			\]
	\end{itemize}

\end{frame}

\begin{frame}
	\frametitle{Likelihood of $\mu$ and $\sigma^2$}

	\hspace*{-5mm}
	\input{normal.likelihood}

	The likelihood of $\mu$ and $\sigma$ when \[
	D = 2.41, 5.37, 5.28, 4.89, 4.40, 4.63, 4.67, 4.52, 1.10, 3.86
	\]
	with $\bar{x}=4.11$ and $s=1.28$.
\end{frame}

\begin{frame}
	\frametitle{Conjugate prior for $\mu$ and $\sigma$}
	\begin{itemize}
		\item A common choice of conjugate prior is the normal/inverse-gamma distribution, also known as the \emph{normal $\times$ scaled inverse-$\chi^2$} distribution.
		\item Thus, our full model is 
			\begin{align*}
				x_i &\sim \textrm{N}(\mu, \sigma^2), \quad\text{for $i\in 1, 2 \cdots n$},\\
				\mu &\sim \textrm{N}(\mu_0, \sigma^2/\kappa_0),\\
				\sigma^2 &\sim \textrm{Inv-}\chi^2(\nu_0, \sigma^2_0).
			\end{align*}
		\item This corresponds to the joint prior density 
			\begin{align*}
				\Prob{\mu,\sigma^2} &= \textrm{N-Inv-}\chi^2(\mu, \sigma^2 \given \mu_0, \kappa_0, \nu_0, \sigma^2_0),\\
				&=\textrm{N}(\mu \given \mu_0, \sigma^2/\kappa_0) \times \textrm{Inv-}\chi^2(\sigma^2\given\nu_0, \sigma^2_0),\\
				&\propto\sigma^{-1}(\sigma^2)^{-(\nu_0/2+1)} e^{\left(-\tfrac{1}{2\sigma^2}\left[ \nu_0\sigma^2_0 + \kappa_0(\mu_0  - \mu)^2 \right]\right)}
			\end{align*}
	\end{itemize}
\end{frame}

\begin{frame}
	\frametitle{The scaled inverse-$\chi^2$ distribution}

	\hspace*{-5mm}
	\input{scaled.inv.chisq}

	The mean is $\tfrac{\nu}{\nu-2}\sigma^2_0$, mode is $\tfrac{\nu}{\nu+2}\sigma^2_0$.

\end{frame}

\begin{frame}
	\frametitle{Joint posterior on $\mu$ and $\sigma$}
	\begin{itemize}
		\item With the normal likelihood and normal/scaled inverse-$\chi^2$ distribution, the joint posterior is:
			\begin{align*}
				\Prob{\mu, \sigma^2 \given D} &\propto \Prob{D\given \mu, \sigma^2} \Prob{\mu,\sigma^2},\\
				\begin{split}
				&\propto\sigma^{-n} e^{\left( -\tfrac{1}{2\sigma^2}\left[n s^2 + n(\bar{x}-\mu)^2\right] \right)}\\
				&\phantom{\propto}\times \sigma^{-1}(\sigma^2)^{-(\nu_0/2+1)} e^{\left(-\tfrac{1}{2\sigma^2}\left[ \nu_0\sigma^2_0 + \kappa_0(\mu_0  - \mu)^2 \right]\right)}
				\end{split}\\
				& = \textrm{N-Inv-}\chi^2(\mu, \sigma^2 \given \mu_n, \kappa_n, \nu_n, \sigma_n^2)
			\end{align*}
		where 
		\begin{align*}
			\mu_n &= \frac{\kappa_0\mu_0 + n\bar{x}}{\kappa_n},\\
			 \kappa_n     &= \kappa_0 + n,\\
			 \nu_n &= \nu_0 + n,\\
			 \sigma_n^2 &= \frac{1}{\nu_n}\left( \nu_0\sigma_0^2 + ns^2 + \frac{n\kappa_0}{\kappa_0 + n}(\mu_0 - \bar{x})^2\right).
		\end{align*}
	\end{itemize}
\end{frame}


\begin{frame}
	\frametitle{Marginal posterior of $\sigma^2$}

	\begin{itemize}
		\item The marginal posterior of $\sigma^2$ is obtained by integrating over $\mu$:
			\begin{align*}
				\Prob{\sigma^2 \given D} &= \int \Prob{\sigma^2, \mu\given D}\ d\mu,\\
				&= \int \textrm{N-Inv-}\chi^2(\mu, \sigma^2 \given \mu_n, \kappa_n, \nu_n, \sigma_n^2) d\mu,\\
				&\propto (\sigma^2)^{-(\nu_n/2+1)}e^{\left(-\frac{1}{2\sigma^2}[\nu_n\sigma_n^2]\right)},\\
				&= \textrm{Inv-}\chi^2(\sigma^2\given\nu_n, \sigma^2_n)
			\end{align*}
	\end{itemize}
\end{frame}

\begin{frame}
	\frametitle{Marginal posterior of $\mu$}

	\begin{itemize}
		\item The marginal posterior of $\mu$ is obtained by integrating over $\sigma^2$:
			\begin{align*}
				\Prob{\mu \given D} &= \int \Prob{\sigma^2, \mu\given D}\ d\sigma^2,\\
				&= \int \textrm{N-Inv-}\chi^2(\mu, \sigma^2 \given \mu_n, \kappa_n, \nu_n, \sigma_n^2) d\sigma^2,\\
				&\propto \left[ 1 + \frac{\kappa_n}{\nu_n \sigma_n^2}\left(\mu-\mu_n\right)^2\right]^{-(\nu_n+1)/2},\\
				&= \textrm{t}_{\nu_n}\left(\mu \given \mu_n, \sigma^2_n/\kappa_n\right).
			\end{align*}
		which is a (non-standard) t-distribution with $\nu_n$ degrees of freedom, location $\nu_n$ and scale $\sigma^2_n/\kappa_n$.
	\end{itemize}
\end{frame}

\begin{frame}
	\frametitle{Marginal posterior density of $\mu$}
	\begin{itemize}
		\item To infer the marginal probability of $\mu$, we must integrate over the \emph{nuisance} variable $\sigma^2$:
			\begin{align*}
				\Prob{\mu\given D} &= \int \Prob{\mu, \sigma^2 \given D}\ d\sigma^2,\\
			\intertext{which can be done by}
			&= \int \overbrace{\Prob{\mu\given \sigma^2, D}}^{\substack{\text{conditional}\\\text{posterior}}}
			\underbrace{\Prob{\sigma^2\given D}}_{\substack{\text{marginal}\\\text{posterior}}}\ d\sigma^2.
			\end{align*}
		\item This demonstrates that the posterior distribution $\Prob{\mu\given D}$ is an infinite mixture of normal distributions.
	\end{itemize}
\end{frame}


\begin{frame}
	\frametitle{Marginal posterior density of $\mu$}
	\begin{itemize}
		\item If we fix $\sigma^2$ at any value and, using the previous prior on $\mu$, i.e. $\mu \sim \textrm{N}(\mu_0, \sigma^2/\kappa_0)$, then 
			\begin{align*}
				\Prob{\mu \given D, \sigma^2} &\propto \Prob{D\given\mu, \sigma}\Prob{\mu\given \mu_0, \sigma^2/\kappa_0},\\
				&= \textrm{N}(\mu_n, \sigma^2/\kappa_n),
			\end{align*}
			where 
			\[
				\mu_n = \frac{\tfrac{\kappa_0}{\sigma^2}\mu_0 + \tfrac{n}{\sigma^2}\bar{x}}{\tfrac{\kappa_0}{\sigma^2} + \tfrac{n}{\sigma^2}},\quad
				\sigma^2/\kappa_n = \frac{1}{\tfrac{\kappa_0}{\sigma^2} + \tfrac{n}{\sigma^2}}.
			\]
		\item Integrating over $\sigma^2$, when $\sigma^2$ has a scaled inverse $\chi^2$ distribution with parameters $\nu_n, \sigma^2_n$ leads to a non-standard t-distribution
			leads to $\textrm{t}_{\nu_n}\left(\mu \given \mu_n, \sigma^2_n/\kappa_n\right)$ as before.
	\end{itemize}
\end{frame}

\begin{frame}
	\frametitle{Posterior predictive distribution}
	\begin{itemize}
		\item Given our $n$ observed values
			\[	D = 2.41, 5.37, 5.28, 4.89, 4.40, 4.63, 4.67, 4.52, 1.10, 3.86, \]
			and our assumed probabilistic generative model, what is \emph{next} value that we predict?
		\item In other words, what is
			\[
				\Prob{x_{n+1} \given D} = \int\int\Prob{x_{n+1}\given \mu, \sigma^2}\Prob{\mu, \sigma^2\given D} d\mu d\sigma^2
			\]
		\item This is again a t-distribution:
			\[
				\textrm{t}_{\nu_n}\left(\mu \given \mu_n, \left(1 + \tfrac{1}{\kappa_n}\right)\sigma^2_n\right)
			\]
	\end{itemize}
\end{frame}

\begin{frame}
	\frametitle{Inferences concerning differences of normal means}
	\begin{itemize}
		\item Let us assume that we have observed two sets of data as follows:
			\[x_1, x_2 \cdots x_n \iidsim N(\mu_{x}, \sigma^2), \quad y_1, y_2 \cdots y_n \iidsim N(\mu_{y}, \sigma^2).\]

		\item In other words, both $x_1, x_2 \cdots x_n$ and $y_1, y_2 \cdots$ are independently and identically normally distributed, with different means $\mu_x$, $\mu_y$ but with a common variance $\sigma^2$.
		\item In general, we know neither $\mu_x$, $\mu_y$ nor $\sigma^2$.
		\item In such a situation, a common problem we are faced with is to infer the difference
			\[ \mu_x - \mu_y,\]
			given $D=\{x_1, x_2 \cdots x_n, y_1, y_2 \cdots y_n\}$, while integrating over the nuisance variable $\sigma^2$, i.e.,
			\[
				\Prob{\mu_x - \mu_y \given D}.
			\]
	\end{itemize}
\end{frame}

\begin{frame}
	\frametitle{Posterior inference of differences of means}
	\begin{itemize}
		\item Inference of the differences of means is a special kind of posterior predictive inference:
			\[
				\int \Prob{\mu_x - \mu_y \given \mu_x, \mu_y, \sigma^2}\Prob{\mu_x, \mu_y, \sigma^2\given D} d\mu_xd\mu_yd\sigma^2.
			\]
		\item We can approach this problem using the methods we have used to deal with inference in the univariate normal distribution.
		\item We begin by assuming, as before,
			\begin{align*}
				%\Prob{\mu_x, \mu_y, \sigma^2} \propto \tfrac{1}{\sigma^2}
				\mu_x &\sim \textrm{N}(\mu_0, \sigma^2/\kappa_0),\\
				\mu_y &\sim \textrm{N}(\mu_0, \sigma^2/\kappa_0),\\
				\sigma^2 &\sim \textrm{Inv-}\chi^2(\nu_0, \sigma^2_0).
			\end{align*}

	\end{itemize}
\end{frame}

\begin{frame}
	\frametitle{Posterior inference of differences of means}
	\begin{itemize}
		\item Given that
			\[ \Prob{\mu_x, \mu_y, \sigma^2 \given D}  = \Prob{\mu_x, \mu_y, \given \sigma^2, D} \Prob{\sigma^2 \given D}\]

			\item From previous results, we see that 
			\begin{align*}
				\Prob{\mu_x, \mu_y, \given \sigma^2, D} &= \Prob{\mu_x \given \sigma^2, D} \Prob{\mu_y\given \sigma^2, D},\\
								        &= N(\mu_n^x, \sigma^2/\kappa^x_n) N(\mu_n^y, \sigma^2/\kappa^y_n) 
			\end{align*}
			where $\mu_n^x$, $\mu_n^y$, $\kappa^x_n$ and $\kappa^y_n$ are defined analogously with previous results.
	\end{itemize}
\end{frame}

\begin{frame}
	\frametitle{Posterior inference of differences of means}
	\begin{itemize}
		\item As such,
			\begin{align*}
				&\int \Prob{\mu_x - \mu_y \given \mu_x, \mu_y, \sigma^2}\Prob{\mu_x, \mu_y \given \sigma^2, D} d\mu_xd\mu_y\\
				&= N(\mu_n^x - \mu_n^y, \sigma^2/\kappa^x_n + \sigma^2/\kappa^y_n).
				\intertext{For simplicity, let's assume $\kappa_0 \doteq 0$. In that case,}
				&= N(\bar{x}- \bar{y}, \tfrac{\sigma^2}{n} + \tfrac{\sigma^2}{m}).
			\end{align*}
		\item Integrating over $\sigma^2$ as before leads to 
			\begin{align*}
				&\int  N(\bar{x}- \bar{y}, \tfrac{\sigma^2}{n} + \tfrac{\sigma^2}{m}) \Prob{\sigma^2 \given D}\\
				&=  \textrm{t}_{\nu_n}\left(\bar{x}-\bar{y}, \sigma^2_n\left(\tfrac{1}{n}+\tfrac{1}{m}\right)\right) 
			\end{align*}
			

	\end{itemize}
\end{frame}

\begin{frame}
	\frametitle{Posterior inference of differences of means}
	\begin{itemize}
		\item In the case of the non-informative prior case, i.e.,
			\[\Prob{\mu_x, \mu_y, \sigma^2} \propto \tfrac{1}{\sigma^2},\]
		the marginal posterior over $\mu_x - \mu_y$ is
		\begin{align*}
			\Prob{\mu_x-\mu_y \given D} = \textrm{t}_{\nu}\left(\bar{x}-\bar{y}, s^2\left(\tfrac{1}{n}+\tfrac{1}{m}\right)\right)
		\end{align*}
		where
		\[ \nu = n + m -2,\quad s^2 = \frac{(n_x-1)s^2_x + (n_y-1)s^2_y}{n_x + n_y - 2}.\]
	\item This is an identical distribution to the sampling distribution of the difference of the means.
	\end{itemize}
\end{frame}

%\begin{frame}
%	\frametitle{Bayesian linear regression}
%	\begin{itemize}
%		\item If our data are
%			\[D = (y_1, x_1), (y_2, x_2) \cdots (y_n, x_n),\]
%		where
%		\[
%		y_i = \beta_0 + \beta_1 x_i + \varepsilon_i, \quad \varepsilon \sim N(0, \sigma^2),
%		\]
%		we generally aim to infer
%		\[
%			\Prob{\beta_0, \beta_1 \given D}.
%		\]
%	\item In this problem, it is convenient to consider the non informative prior 
%		\[\Prob{\beta_0, \beta_1, \sigma^2} \propto \tfrac{1}{\sigma^2}.\]
%	\end{itemize}
%\end{frame}
%

\end{document}
