\documentclass{slides}
\usepackage{xspace}


\title[Bayesian Inference]{Introduction to Bayesian Inference}
\author[Andrews]{Mark Andrews}
\date{}

\begin{document}
{
	\begin{frame}
		\titlepage
	\end{frame}
}


\begin{frame}
\frametitle{Inference in a Bernoulli Distribution}

\begin{quotation} \ldots Polish mathematicians Tomasz Gliszczynski and
	Waclaw Zawadowski\ldots spun a Belgian one euro coin 250 times,
	and found it landed heads up 140 times \ldots When tossed 250
	times, the one euro coin came up heads 139 times and tails 111.
	\ldots 
\end{quotation}

\flushright{\emph{The Guardian}, January 4, 2002\footnote{ See
	\texttt{http://bit.ly/1BOKu9b} for original story and
	\texttt{http://bit.ly/1BOKx4Q} for discussion.}}

	\begin{itemize}
		\item A sample of $n=250$ coin tosses can be modelled as $n$ independent and identically dsibution Bernoulli random variables with parameter $\theta$.
		\item In other words, our probabilitic model is
			\[
				x_i \sim \textrm{Bernoulli}(\theta),\quad\text{for $i \in \{1,2\ldots n\}$.}
			\]
			and we would like to inference the probable values of $\theta$ given an observation of $m=139$ (or $m=140$, etc.).
	\end{itemize}
\end{frame}

\begin{frame}
	\frametitle{Inference in a Bernoulli Distribution}
	\begin{itemize}
		\item The likelihood of the observed outcomes of the $n$ coins is
			\begin{align*}
				\Prob{x_1, x_2\cdots x_n \given \theta} &= \prod_{i=1}^n \Prob{x_i \given \theta},\\
				& = \prod_{i=1}^n \theta^{x_i} (1-\theta)^{1-x_i},\\
				& = \theta^{m} (1-\theta)^{n-m}.
				\label{eq:bernoulli-likelihood}
			\end{align*}
		\item This is identical to a binomial distribution likelihood function.
	\end{itemize}
\end{frame}

\begin{frame}
	\frametitle{Inference in a Bernoulli Distribution}
	\input{binomial.likelihood}
		The likelihood function for $n=250$ and $m=139$.
\end{frame}
\begin{frame}
	\frametitle{Inference in a Bernoulli Distribution}
	\begin{itemize}
		\item The probabilistic generative model of the Euro coin toss data is as follows:
			\begin{itemize}
				\item The coin's bias corresponds to the fixed but unknown value of the parameter $\theta$ of a Bernoulli random variable.
				\item The observed outcomes $x_1, x_2 \cdots x_n$ are $n$ iid samples from $\textrm{Bernoulli}(\theta)$.
			\end{itemize}
		\item This generative model can be extended by assuming that $\theta$ is itself drawn from a prior distribution $\Prob{\theta}$:
			\begin{align*}
				\theta &\sim \Prob{\theta},\\
				x_i &\sim \textrm{Bernoulli}(\theta),\quad\text{for $i \in \{1,2\ldots n\}$.}
			\end{align*}
		\item In other words, we assume that a value for $\theta$ was randomly drawn from $\Prob{\theta}$ and then $n$ binary variables were sampled from $\textrm{Bernoulli}(\theta)$.
	\end{itemize}
\end{frame}

\begin{frame}
	\frametitle{Conjugate Priors}
	\begin{itemize}
		\item For a given likelihood function, a \emph{conjugate prior} distribution is a prior probability distribution that leads to a posterior distribution of the same parameteric family.
		\item Using conjugate priors allows Bayesian inference and other probabilistic calculations to be performed analytically.
		\item Only a small subset of probabilistic models have conjugate priors.
		\item However, conjugate priors play a vital role in Monte Carlo methods like Gibbs sampling even in complex models.
	\end{itemize}
\end{frame}

\begin{frame}
	\frametitle{The beta distribution}
	\begin{itemize}
		\item For the binomial likelihood function
\[
	\theta^m (1-\theta)^{n-m}	
\]
a conjugate prior is the beta distribution
\[
	\textrm{Beta}(\theta \given \alpha, \beta) = \frac{\Gamma(\alpha + \beta)}{\Gamma(\alpha)\Gamma(\beta)} \theta^{\alpha-1}(1 - \theta)^{\beta-1}.
\]
\item The \emph{normalizing constant} term is
	\[
		\frac{\Gamma(\alpha + \beta)}{\Gamma(\alpha)\Gamma(\beta)} = \frac{1}{B(\alpha,\beta)},
	\]
	where $B(\alpha,\beta)$ is the beta function:
	\[
		B(\alpha,\beta) = \int \theta^\alpha (1-\theta)^{\beta-1}\ d\theta.
	\]
	\end{itemize}
\end{frame}
\begin{frame}
	\frametitle{The beta distribution}
	\input{beta.3.5.distribution}
	The beta distribution with $\alpha=3$ and $\beta=5$.
\end{frame}
\begin{frame}
	\frametitle{Posterior distribution}
	\begin{itemize}
		\item Denoting the observed data by $D = (n, m)$, with the beta prior, the posterior distribution is 

		\begin{align*}
			\Prob{\theta\given D, \alpha,\beta} &= \frac{\Prob{D\given\theta}\Prob{\theta\given\alpha,\beta}}{\int\Prob{D\given\theta}\Prob{\theta\given\alpha,\beta}\ d\theta},\\
			&\propto \overbrace{\theta^{m} (1-\theta)^{n-m}}^{\text{likelihood}} \times \overbrace{\theta^{\alpha-1}(1 - \theta)^{\beta-1}}^{\text{prior}},\\
			&\propto \theta^{m + \alpha-1}(1 - \theta)^{n-m+\beta-1},\\
			&= \textrm{Beta}(m + \alpha, n - m + \beta).
		\end{align*}
		where the normalizing constant is the reciprocal of the beta function 
		\begin{align*}
			\frac{\Gamma(m + \alpha)\Gamma(n - m + \beta)}{\Gamma(n + \alpha+\beta)} &= \int \theta^{\alpha + m - 1} (1-\theta)^{ \beta + n - m - 1}\ d\theta.\\
			&=B(\alpha + m, \beta + n - m).
		\end{align*}

	\end{itemize}
\end{frame}

\begin{frame}
	\frametitle{Posterior distribution}
	\begin{itemize}
		\item For our Euro coin example, our observed data are $n=250$ and $m=139$.
		\item A noninformative uniform prior on $\theta$ is $\textrm{Beta}(\alpha=1,\beta=1)$.
		\item With this prior, the posterior distribution is 
			\begin{align*}
				\textrm{Beta}(m + \alpha, n - m + \beta) &= \textrm{Beta}(139 + 1, 250 - 139 + 1),\\
				&= \textrm{Beta}(140, 112)
			\end{align*}
	\end{itemize}
\end{frame}
\begin{frame}
	\frametitle{Posterior distribution}
	\input{beta.140.112.distribution}
	The posterior distribution when $n=250$, $m=139$, $\alpha=1$ and $\beta=1$.
\end{frame}

\begin{frame}
	\frametitle{Summarizing the posterior distribution}
	\begin{itemize}
		\item The mean, variance and modes of any beta distribution are as follows:
			\begin{align*}
				\langle \theta \rangle &=\frac{\alpha}{\alpha+\beta} ,\\
				\mathrm{V}(\theta) &= \frac{\alpha\beta}{(\alpha+\beta)^2(\alpha+\beta+1)},\\
				\textrm{mode}(\theta) &= \frac{\alpha-1}{\alpha+\beta-2}.
			\end{align*}
		\item Thus, in our case of $\textrm{Beta}(140, 112)$, we have
			\begin{align*}
				\langle \theta \rangle &=  0.5556,\\
				\mathrm{V}(\theta) &= 0.001, \quad\textrm{sd}(\theta) = 0.0312,\\
				\textrm{mode}(\theta) &= 0.556.
			\end{align*}

	\end{itemize}
\end{frame}


\begin{frame}
	\frametitle{High posterior density (\hpd) intervals}
	\begin{itemize}
		\item \hpd intervals provide ranges that contain specified probability mass. For example, the 0.95 \hpd interval is the range of values that contain 0.95 of the probability mass of the distribution.
		\item The $\varphi$ \hpd interval for the probability density function $\Prob{x}$ is computed by finding a probability density value $p^*$ such that 
		\[
			\Prob{\{x \colon \Prob{x} \geq p^*\}} = \varphi.
		\]
		\item In other words, we find the value $p^*$ such that the probability mass of the set of points whose density is greater than than $p^*$ is exactly $\varphi$. 
		\item In general, the \hpd is not trivial to compute but in the case of symmetric distributions, it can be easily computed from the cumulative density function.
	\end{itemize}
\end{frame}

\begin{frame}
	\frametitle{The 0.95 \hpd interval}
	\input{beta.140.112.hpd}
	The posterior distribution, with its $0.95$ \hpd, when $n=250$, $m=139$, $\alpha=1$ and $\beta=1$. In this case, the \hpd interval is $(0.494, 0.617)$.
\end{frame}

\begin{frame}
	\frametitle{Posterior predictive distribution}

	\begin{itemize}
		\item Given that we have observed $m$ heads in $n$ coin tosses, what is the probability that the \emph{next} coin toss is heads.
		\item This is given by the \emph{posterior predictive} probability that $x=1$:
			\begin{align*}
				\Prob{x=1\given D, \alpha, \beta} &= \int \Prob{x=1 \given \theta} \overbrace{\Prob{\theta\given D, \alpha, \beta}}^{\text{Posterior}}\ d\theta,\\
				&= \int \theta \times \Prob{\theta\given D, \alpha, \beta} \ d\theta,\\
				&= \left\langle \theta \right\rangle,\\
				&= \frac{\alpha+m}{\alpha+\beta+n}.
			\end{align*}
		\item Thus, given 139 heads in 250 tosses, the predicted probability that the next coin will also be heads is $\approx 0.5556$.
	\end{itemize}
\end{frame}
\begin{frame}
	\frametitle{Marginal likelihood}	
	\begin{itemize}
		\item The posterior distribution is 
		\[
			\Prob{\theta\given D, \alpha,\beta}
			= \frac{\overbrace{\Prob{D\given\theta}}^{\text{Likelihood}}\overbrace{\Prob{\theta\given\alpha,\beta}}^{\text{Prior}}}
			{\underbrace{\int\Prob{D\given\theta}\Prob{\theta\given\alpha,\beta}\ d\theta}_{\text{Marginal likelihood}}}.
		\]
		where the \emph{marginal likelihood} gives the likelihood of the model given the observed data:
		\[
			\int\Prob{D\given\theta}\Prob{\theta\given\alpha,\beta}\ d\theta \defeq \Prob{D\given \alpha,\beta}.
		\]
	\item In this example, it has a simple analytical form:
		\[
			\Prob{D\given \alpha,\beta} = B(\alpha+m, \beta+n-m) = \frac{\Gamma(m + \alpha)\Gamma(n - m + \beta)}{\Gamma(n + \alpha+\beta)}.
		\]


	\end{itemize}
\end{frame}

\begin{frame}
	\frametitle{Model comparison}
	\begin{itemize}
		\item Given $D$, we can compare the probability of model $M_1$ relative to model $M_0$ as follows:
			\begin{align*}
				\frac{\Prob{M_1\given D}}{\Prob{M_0\given D}} = 
				\underbrace{\frac{\Prob{D\given M_1}}{\Prob{D\given M_0}}}_{\text{Bayes factor}} \times
				\underbrace{\frac{\Prob{M_1}}{\Prob{M_0}}}_{\text{Priors odds}}. 
			\end{align*}
		\item When both models are equally probable a priori, then the relative posterior probabilities is determined by the Bayes factor.
		\item We can compare our model $M_1$, i.e. with $\alpha = \beta = 1$, with the $M_0$ model that $\theta = \frac{1}{2}$.
			\begin{align*}
				\frac{\Prob{D\given M_1}}{\Prob{D\given M_0}} = \frac{
				\int\Prob{D\given\theta}\Prob{\theta\given\alpha=1,\beta=1}\ d\theta 
				}{
					\int\Prob{D\given\theta} \delta(\theta-\frac{1}{2})\ d\theta.
				}
			\end{align*}

		\item This effectively compares a model that assumes a
			completely random coin machine, i.e., coin biases are
			generated according to $\textrm{Beta}(\alpha=1,
			\beta=1)$, to a completely perfect coin machine, i.e.,
			coin biases are generated according to
			$\delta(\theta-\frac{1}{2})$.  \end{itemize}

	\end{frame} 
	
\begin{frame}
	\frametitle{Model comparison}
	\begin{itemize}
		\item We can compare our model $M_1$, i.e. with $\alpha = \beta = 1$, with the $M_0$ model that $\theta = \frac{1}{2}$.
			\begin{align*}
				\frac{\Prob{D\given M_1}}{\Prob{D\given M_0}} &= \frac{
				\int\Prob{D\given\theta}\Prob{\theta\given\alpha=1,\beta=1}\ d\theta 
				}{
					\int\Prob{D\given\theta} \delta(\theta-\frac{1}{2})\ d\theta.
				},\\
				&= \frac{\Gamma(\alpha+m)\Gamma(\beta+n-m)}{\Gamma(\alpha+\beta+n)} \Big/ \tfrac{1}{2}^m (1-\tfrac{1}{2})^{n-m},\\
				&= \frac{m!(n-m)!}{(n+1)!} \big/ \tfrac{1}{2^n}.\\
				\intertext{If $n=250$, $m=139$, then}
				&= \frac{139!111!}{251!} \big/ \tfrac{1}{2^{250}} = 0.38.
			\end{align*}
		\item This is a factor of $2.65$ in favour of the unbiased coin hypothesis.
		\item Note that the classical statistics null hypothesis test gives a p-value of $p=0.0875$.
	\end{itemize}
	\let\thefootnote\relax\footnote{If $n$ is a positive integer $\Gamma(n) = (n-1)!$.}
	\end{frame}

\end{document}
